\documentclass[graphics, twocolumn, usenatbib]{mn2e}
\usepackage[utf8]{inputenc}
\usepackage{times}
\usepackage{natbib} 
\usepackage{epsfig} 
\usepackage{graphicx} 
\usepackage[dvipsnames]{xcolor}
%\usepackage{deluxetable} 
\usepackage{aas_macros} 
\usepackage{amssymb}
\usepackage{amsmath}
\usepackage[title]{appendix}
\usepackage{hyperref}	% Hyperlinks
\hypersetup{colorlinks=true,linkcolor=blue,citecolor=blue,filecolor=blue,urlcolor=blue}
\usepackage[caption=false]{subfig}
\usepackage{soul}                          % provides \hl{} for highlighting
\usepackage{ulem} \normalem 
\usepackage{xifthen}
\newcommand{\lya}{Lyman-$\alpha$~}
\newcommand{\gad}{\textsc{Gadget-2~}}
\newcommand{\enzo}{\texttt{Enzo~}}
\newcommand{\enzoc}{\texttt{Enzo}}
\newcommand{\yt}{\texttt{yt~}}
\newcommand{\ytc}{\texttt{yt}}
\newcommand{\cloudy}{\texttt{Cloudy~}}
\newcommand{\grackle}{\texttt{Grackle-2.1~}}
\newcommand{\gracklec}{\texttt{Grackle-2.1}}
\newcommand{\enzolat}{\texttt{Enzo-2.3}}
\newcommand{\cosmos}{\textsc{Cosmos~}}

\newcommand{\darwin}{\textsc{Darwin~}}
\newcommand{\enzoamr}{\texttt{Enzo(AMR)~}}
\newcommand{\enzost}{\texttt{Enzo(static)~}}
\newcommand{\lyb}{Lyman-$\beta$~} 
\newcommand{\eg}{{\it e.g.~}}
\newcommand{\kms} {km $\rm{s^{-1}}$}
\newcommand{\mpch} {\rm $h^{-1}$ Mpc\,\,} 
\newcommand{\kpch} {\rm $h^{-1}$ kpc\,\,} 
\newcommand{\msolar} {$\rm{M_{\odot}}~$}
\newcommand{\msolarc} {$\rm{M_{\odot}}$}
%\newcommand{\msolaryr} {$\rm{M_{\odot}/yr}~$}
%\newcommand{\msolaryrc} {$\rm{M_{\odot}/yr}$}
\newcommand{\msolaryr} {$\rm{M_{\odot}~yr^{-1}}~$}
\newcommand{\msolaryrc} {$\rm{M_{\odot}~yr^{-1}}$}
\newcommand{\lsolar} {$\rm{L_{\odot}}~$}
\newcommand{\lsolarc} {$\rm{L_{\odot}}$}
\newcommand{\zsolar} {$\rm{Z_{\odot}}~$}
\newcommand{\zsolarc} {$\rm{Z_{\odot}}$}
\newcommand{\rsolar} {$\rm{R_{\odot}}~$}
\newcommand{\molH} {$\rm{H_2}$~}
\newcommand{\molHc} {$\rm{H_2}$}
\newcommand{\J} {$\rm{10^{-21}\ erg\ cm^{-2}\ s^{-1}\ Hz^{-1}\ sr^{-1}}$}
\newcommand{\inten} {$\rm{ erg\ cm^{-2}\ s^{-1}\ Hz^{-1}\ sr^{-1}}$~}
\newcommand{\JU} {$\rm{ erg\ cm^{-2}\ s^{-1}\ Hz^{-1}\ sr^{-1}}$}
\newcommand{\JLW} {J$_{\rm LW}$}
\newcommand{\healpix} {\texttt{HEALPix~}}
\newcommand{\smartstar} {\texttt{SmartStar~}}
\newcommand{\smartstars} {\texttt{SmartStars~}}
\newcommand{\smartstarsc} {\texttt{SmartStars}}
\newcommand{\smartstarc} {\texttt{SmartStar}}
\newcommand{\rarepeak} {\textit{Rarepeak~}}
\newcommand{\rarepeakc} {\textit{Rarepeak}}
\newcommand{\normal} {\textit{Normal~}}
\newcommand{\normalc} {\textit{Normal}}
\newcommand{\void} {\textit{Void~}}
\newcommand{\voidc} {\textit{Void}}
\newcommand{\ha} {\texttt{HaloA~}}
\newcommand{\hb} {\texttt{HaloB~}}
\newcommand{\hac} {\texttt{HaloA}}
\newcommand{\hbc} {\texttt{HaloB}}

\def\mgh#1{{\bf MH:  #1}}
\def\jr#1{{\color{blue} \bf JR:  #1}}
\def\pj#1{{\bf PJ:  #1}}
\newcommand{\jhw}[1]{{\color{Maroon} (\bf JHW: #1)}}
\newcommand{\note}[1]{{\noindent\hspace{-3em}\bf\color{Plum}$\longrightarrow$ \quad{#1}}}
\newcommand{\delete}[1]{{\color{red}{\sout{#1}}}}
\newcommand{\change}[2][]{%
\ifthenelse{\isempty{#2}}{{\color{ForestGreen}{#1}}}%
{{\color{RedOrange}\sout{#1}}{\color{ForestGreen}{ #2}}}%
}


\def\etal{{\it et al.}~}

\begin{document}
\title{Massive Star Formation in Metal Free Galaxies at High Redshift}
\author[J. A. Regan, J. H. Wise,  T. E. Woods, T.P. Downes, B.W. O'Shea \&  M.L. Norman]{John A. Regan$^{1,2}$\thanks{E-mail:john.regan@mu.ie},
  John H. Wise$^{3}$, Tyrone E. Woods$^{4}$, Turlough P. Downes$^{2}$, \newauthor Brian W. O'Shea$^{5,6,7,8}$ \& Michael L. Norman$^9$\\
  $^1$Department of Theoretical Physics, Maynooth University, Ireland\\
  $^2$Centre for Astrophysics \& Relativity, School of Mathematical Sciences, Dublin City University, Glasnevin, D09 W6Y4, Ireland\\
  $^3$Center for Relativistic Astrophysics, Georgia Institute of Technology, 837 State Street, Atlanta, GA 30332, USA\\
  $^4$National Research Council of Canada, Herzberg Astronomy \& Astrophysics Research Centre, 5071 West Saanich Road, Victoria, BC V9E 2E7, Canada\\
  $^5$Department of Computational Mathematics, Science, and Engineering, Michigan State University, MI, 48823, USA\\    
  $^6$Department of Physics and Astronomy, Michigan State University,MI, 48823, USA\\
  $^7$National Superconducting Cyclotron Laboratory, Michigan State, University, MI, 48823, USA\\
  $^8$Joint Institute for Nuclear Astrophysics - Center for the Evolution of the Elements, USA\\
  $^9$Center for Astrophysics and Space Sciences, University of California, San Diego, 9500 Gilman Dr, La Jolla, CA 92093\\}

\pubyear{2020}
\label{firstpage}
\pagerange{\pageref{firstpage}--\pageref{lastpage}}
\maketitle

\begin{abstract}
  We investigate the ab-initio formation of super-massive stars in a pristine atomic cooling halo.
  The halo is extracted from a larger self-consistent parent simulation. The halo remains metal-free
  and star formation is suppressed due to a combination of dynamical heating from mergers and a
  mild ($J_{\rm LW} \sim 2 - 10 \ J_{21}$(z)) Lyman-Werner (LW) background.
  We find that more than 20 very massive stars form with stellar
  masses greater than 1000 \msolarc. The most massive star has a stellar mass of over 6000  \msolarc.
  However, accretion onto all stars declines significantly after the first 10 - 20 kyr of evolution
  as the surrounding material is accreted and the turbulent nature of the gas causes the stars to
  move to lower density regions. We post-process the impact of ionising radiation from the stars and
  find that ionising radiation is not a limiting factor when considering SMS formation
  and growth. Rather the birth environments are highly turbulent and a steady accretion flow is not
  maintained within the timescale (2 Myrs) of our simulations. \jhw{Is this similar to present-day star formation?} As the massive stars end their lives
  as direct collapse black holes this will seed these embryonic haloes with a population of black
  holes with masses between 1000 \msolar and 10,000 \msolarc.
  Afterwards they may sink to the centre of the haloes, eventually coalescing to form larger
  intermediate mass black holes whose in-situ mergers will be detectable by LISA. 
\end{abstract}

\begin{keywords}
  Early Universe, Supermassive Stars, Star Formation, First Galaxies, Numerical Methods
\end{keywords}

\section{Introduction} \label{Sec:Introduction}
Supermassive stars (SMSs) with masses between $10^4$ and $10^5$ \msolar have over the past few decades been invoked \citep{Rees_1978, Begelman_1978, Begelman_2006,
  Begelman_2008, Latif_2016a, Woods_2018} to explain the existence of supermassive black holes (SMBHs) at the centres of massive galaxies \citep{Fan_06, Kormendy_2013}.
The pathways to forming a SMBH are thus far unknown with a number of theoretical models proposing to explain their existence. \\
\indent Perhaps the
simplest explanation is to invoke the black holes left over from the first 
generation of stars as seeds for SMBHs. The first generation of (metal-free) 
stars are referred to as Population III (PopIII) stars and according to 
current theoretical models \citep[e.g.][]{Turk_2009, Clark_2008, Hirano_2014, Stacy_2016} the initial mass function should be
top heavy with a characteristic mass of tens of solar masses. However, PopIII remnant black holes are expected to form in low
density environments \citep{Whalen_2004, OShea_2005b, Milosavljevic_2009} and are not expected to accrete substantially, at least not 
initially \citep{Alvarez_2009, Smith_2018}. PopIII stars are therefore not seen as good candidates to explain the existence
of SMBHs at least not without invoking super-Eddington accretion scenarios which can boost their initial 
seed masses by an order of magnitude or two over a short period
\citep{Lupi_2014, Pacucci_2015a, Sakurai_2016a,Inayoshi_2016, Pacucci_2017, Inayoshi_2018}.\\
\indent SMSs provide an alternative path to forming a SMBH by giving the seed black hole a
head-start compared to a black hole formed from a PopIII remnant. Under a SMS formation scenario
the accretion rate onto the protostar must exceed a critical threshold thought to be around
0.04 \msolaryr \citep{Sakurai_2016}. When
this threshold accretion rate is reached and maintained the stellar radius inflates reducing
its surface temperature to approximately 5000 K
and making the star resemble a red giant star
\citep{Omukai_2003, Hosokawa_2012, Hosokawa_2013, Woods_2017}. However, the SMS must continue to
accrete above this threshold rate. If the accretion rate falls below the critical
rate the star contracts to the main sequence and becomes a hyper-luminous PopIII
star with a mass set approximately by the mass at which the accretion rate dropped.
When the accretion rate is maintained the star grows rapidly but emits
only weak radiative feedback with the spectrum of the emitted radiation peaking below
the hydrogen limit \citep{Woods_2018}. \\
\indent As discussed, the key requirement for
forming a SMS is that the mass accretion rate onto the star exceeds approximately 0.04 \msolaryrc,
however, a sufficient baryon reservoir is also required and furthermore the metallicity of the gas
being accreted should be below $10^{-3}$ \zsolar \citep{Chon_2020}. For these reasons
metal-poor (i.e., Z $\lesssim 10^{-3}$ \zsolarc) atomic cooling haloes are seen as the most 
promising candidates in which to form SMS. Haloes which have higher levels of 
metal-enrichment (Z $> 10^{-3}$ \zsolarc) may also be viable candidates for 
SMS formation in the early universe if metal mixing is inhomogeneous \citep{Regan_2020a}. Atomic cooling
haloes which provide the above requirements for SMS formation were recently investigated by \cite{Wise_2019}
and \cite{Regan_2020}. In particular \cite{Wise_2019} found that the combination 
of a mild Lyman-Werner background combined with the impact of dynamical heating effects due to
minor and major mergers can suppress star formation until a halo crosses the atomic cooling threshold.
These haloes are therefore predominantly metal-poor (with any metal enrichment coming externally), have large 
baryon reservoirs and suppressed \molH content due to the LW radiation fields. 
In this study we build on the previous works cited above. \\
\indent The goal of this study is to model the formation and evolution of (super-)massive star
  formation in haloes that are exposed to moderate LW backgrounds, which when combined with the effects
  of dynamical heating can suppresses star formation below the atomic cooling limit.
  Such haloes have recently been identified in the Renaissance datasets \citep{Wise_2019, Regan_2020}
  as promising candidates for forming (super-)massive stars.  To pursue this research we re-simulate
  two haloes from the original Renaissance simulations using the zoom technique.
  
  \note{From this point onwards, this information is probably best stated in the methods section.  If you choose to move it, we probably want to summarize the reasons of the re-simulation rather than an analysis of the Renaissance Simulations.}
  
  We designate
  these haloes as \ha and \hbc. Both haloes were chosen as they exhibited near isothermal
  collapse of their inner core in the original Renaissance datasets as shown by \cite{Regan_2020}.
  They were therefore identified as among the most promising candidates for SMS formation. Both
  haloes were exposed to moderate levels of LW radiation from nearby radiation sources as well as
  constant mergers which dynamically heated the gas within the haloes. To further understand the
  impact of the LW field we re-simulate \hb with no LW radiation. This is done to determine if a
  halo can remain star-free due to only dynamical heating effects or if the LW field 
  remains a critical component. \hac, on the other-hand, is re-simulated with a LW background
  composed of both local source contributions and background contributions. \\
  \indent In the zoom simulations, we find that \ha forms stars with masses greater than 6000 \msolar but that the accretion
  rate onto individual proto-stars always declines as the star's immediate gas supply is depleted. In the
  re-simulation of \hbc, without a LW field, we find that the halo undergoes premature collapse
  (compared to the original case where a LW field of \JLW $\sim 2$ J$_{21}$ existed). In \hb the
  most massive star in the halo has a mass of approximately 173 \msolar. The impact of ionising
  radiation is not considered in these simulations but post-processing of the stellar feedback
  using \cloudy \citep{Ferland_2017} is instead used to gauge the point at which ionising radiation
  would have suppressed further star formation in each halo. \\
  \indent The paper is laid out as follows: In \S \ref{Sec:Methods} we very briefly review the original Renaissance
  simulations as well as discussing the zoom-in simulations. In \S \ref{Sec:Results} we analyse the results of the
  zoom-in simulations. In \S \ref{Sec:GW} we discuss the implications of the results and the connection with
  upcoming gravitational wave observatories. In \S \ref{Sec:Discussion} we summarize our results and outline
  our conclusions.
  
  %%%%%%%%%%%%%%%%%%%%%%%%%%%%%%%%%%%%%FIGURE 1%%%%%%%%%%%%%%%%%%%%%%%%%%%%%%%%%%%%%%%%%%%%%%%%%%%%%%%%%%%
%%%%%%%%%%%Visualisations of haloes at start and end %%%%%%%%%%%%%%%%%
\begin{figure*}
\centering
\begin{minipage}{175mm}      \begin{center} 
\centerline{
\includegraphics[width=0.52\textwidth]{FIGURES/Combined.pdf}
\includegraphics[width=0.52\textwidth]{FIGURES/MassRedshift.pdf}}
\caption{\textit{Left Panel:} The LW background rate imposed on Halo A for the re-simulation. \hb is not shown
  as no LW background is imposed on \hbc. The LW rates
  are composed of local sources, plus a LW background from \citet{Wise_2012b}, extracted from the original
  Renaissance simulations. The local sources dominate over the background by at least an order of magnitude at
  all times. In the original
  Renaissance simulations \ha (and \texttt{HaloB})  remained metal-free and star-free until $z = 15.6$ (and
  $z = 15.0$ respectively). The redshift of original collapse is marked with a green circle.
  A black star marks the redshift at which star formation occurs in \ha in the zoom-in simulations.
  \textit{Right Panel:} The merger history of both \ha and \hb
  as determined from the original Renaissance simulations. Again we mark the redshift of first star formation
  in \ha and \hb as found in the re-simulations with black stars. Note that star formation now occurs significantly
  earlier in \hb because no LW background is imposed. In \hac, on the other hand, star formation occurs 15 Myr after
  the halo was detected in the original (lower resolution) simulations. M$_{\rm atom}$, orange line, is the atomic cooling threshold \citep{Fernandez_2014} and M$_{\rm min,LW}$ is the threshold mass for halo collapse in the presence of an expected LW background at these redshifts \citep{Machacek_2001, OShea_2008}.
  }  \label{Fig:LWHistory}
\end{center} \end{minipage}

\end{figure*}
%%%%%%%%%%%%%%%%%%%%%%%%%%%%%%%%%%%%%%%%%%%%%%%%%%%%%%%%%%%%%%%%%%%%%%%%%%%%%%%%%%%%%%%%%%%%%%%%%%%%%%%%%

\section{Methods} \label{Sec:Methods}


\subsection{Renaissance Simulation Suite} \label{Sec:Renaissance}
\enzo has been extensively used to study the formation of structure in the early universe
\citep{Abel_2002, OShea_2005b, Turk_2012, Wise_2012b, Wise_2014, Regan_2015, Regan_2017}.
\enzo includes a ray tracing scheme to follow the propagation of radiation from
star formation and black hole formation \citep{WiseAbel_2011} as well as a detailed multi-species
chemistry model that tracks the formation and evolution of nine species \citep{Anninos_1997,
  Abel_1997}. In particular the photo-dissociation of \molH is followed, which is a critical
ingredient for determining the formation of the first metal-free stars \citep{Abel_2000}.\\
\indent The original Renaissance simulations were carried out on the Blue Waters supercomputer 
using the adaptive mesh refinement code \enzo\citep{Enzo_2014, Enzo_2019}\footnote{https://enzo-project.org/}.
The datasets that formed the basis for this study were originally derived from a simulation of the
universe in a 40 Mpc on the side box using the WMAP7 best fit cosmology \citep{Komatsu_2011}.
For more details on the Renaissance simulation suite see \cite{Chen_2014}. Here we outline only
the details relevant to this study for brevity. The simulation suite was broken down into
three separate regions, namely the \rarepeak, \normal and \void regions. Each region was simulated with
an effective initial resolution of $4096^3$ grid cells and particles giving a maximum dark matter
particle mass resolution of $2.9 \times 10^4$ \msolarc. Further refinement was allowed throughout
each region up to a maximum refinement level of 12, which corresponded to 19 pc comoving spatial resolution.
Given that the regions focus on different
 overdensities each region was evolved forward in time to different epochs. The \rarepeak region,
 being the most overdense and hence the most computationally demanding at earlier times, was run
 until $z = 15$. The \normal region ran until $z = 11.6$, and the \void region ran until $z = 8$. In all
 of the regions the halo mass function was very well resolved down to M$_{\rm halo} \sim 2 \times 10^6$
 \msolarc. \\
\indent  As noted already in \S \ref{Sec:Introduction}, in \cite{Wise_2019} we examined two metal-free and star-free
 haloes from the final output of the \rarepeak simulation and re-simulated those two haloes at
 significantly higher resolution (maximum spatial resolution, $\Delta x \sim 60$ au) until the
 point of collapse. This re-simulation allowed us to investigate the evolution of the inner halo
 and the mass distribution of the clumps formed. However, no star formation prescription was
 employed during this re-simulation. In \cite{Regan_2020} we subsequently investigated the occurrence
 of metal-free and star-free atomic cooling haloes across all of the Renaissance datasets. We found
 a total of 79 such haloes in the
 \rarepeak outputs and three such haloes in the \normal outputs. None were found in the \void outputs.
 Of the 79 haloes which were metal-free and star free above the atomic cooling limit some
 showed almost ideal isothermal collapse consistent with what has previously been identified
 as ideal conditions for forming SMSs \citep{Inayoshi_2014, Becerra_2015, Latif_2016a,
   Regan_2017, Chon_2017b, Regan_2018b}. Of those haloes which collapsed isothermally we then
 selected two haloes for re-simulation in this study. \\






 \subsection{Re-simulation of pristine atomic cooling haloes} \label{Sec:Resimulation}
In order to make the simulation time tractable, we restricted the mesh refinement to be focused
around the target haloes. In order to do this we first identified the Lagrangian
volume (setting this to three times the virial radius) of the target halo
at the redshift at which it was found ($z = 15.6$ for \ha and $z = 15.0$
 for \hbc) and tagged each dark matter particle within this volume. This ensured
 that we  captured the dynamics of the gas and dark matter both withing and surrounding the halo.
 Upon restarting
 the simulation subsequently at $z = 20$ we then set each of the tagged dark matter particles as
 ``must-refine-particles.'' This meant that any cell, in the simulation, containing one of these
 particles was allowed to refine. Any cells not containing one of these particles could not refine.
 This optimisation focuses the refinement solely onto the target halo (and gas surrounding the halo).
 In addition to the ``must-refine-particle'' refinement criteria,  refinement is updated to be
 also based on the Jeans length of the
 gas with the additional criteria that the Jeans length is always refined by at least 64 cells. \\ 
 \indent Having now optimised the simulations to focus only on the target halo 
 we next look to (dark matter) particle splitting. By splitting the dark matter
 particles we increase the mass resolution of the simulation so as to match the increased spatial resolution.
 Dark matter particles are split in \enzo using the prescription of \cite{Kitsionas_2002} and was previously
 described in \cite{Regan_2015}. By employing particle splitting we reduce the dark matter particle mass
 to $M_{\rm DM} \sim 170$ \msolar inside the target halo and its surrounding Lagrangian volume.
 %%%%%%%%%%%%%%%%%%%%%%%%%%%%%%%%%%%%%FIGURE 2%%%%%%%%%%%%%%%%%%%%%%%%%%%%%%%%%%%%%%%%%%%%%%%%%%%%%%%%%%%
%%%%%%%%%%%Visualisations of haloes at start and end %%%%%%%%%%%%%%%%%
\begin{figure*}
\centering
\begin{minipage}{175mm}      \begin{center} 
\centerline{
\includegraphics[width=0.49\textwidth]{FIGURES/Ionisation.pdf}
\includegraphics[width=0.49\textwidth]{FIGURES/Recombination.pdf}}
\caption{\textit{Left Panel:} The percentage of the inner volume ionised in each halo. For \ha the
    inner volume is set as the central 5pc while for \hb the inner volume is set at 1 pc. In each case
    the centre is defined as the position of the most massive star at that time. After approximately
    1100 kyr \hb has become fully ionised, \ha meanwhile is approximately 46\% ionised after 2000
    kyr. \textit{Right Panel:} The ratio of the ionising photon luminosity and the total integrated
    recombination rate of the gas. For \hb the results are similar to the volume averaged
    calculation - the region is fully ionised after approximately 1100 kyr. However, for \ha the
    results are quite different. The degree of ionisation is much smaller with recombinations easily
    able to dominate ionisations. This is due to the multitude of dense clumps and filaments
    spread throughout \hac. 
  }  \label{Fig:Ionisation}
\end{center} \end{minipage}

\end{figure*}
%%%%%%%%%%%%%%%%%%%%%%%%%%%%%%%%%%%%%%%%%%%%%%%%%%%%%%%%%%%%%%%%%%%%%%%%%%%%%%%%%%%%%%%%%%%%%%%%%%%%%%%%%

 \subsection{The External Radiation Field} \label{Sec:LWRadField}
 With refinement targeted only on a single halo (and its progenitors) star formation in the surrounding galaxies is therefore neglected. In order to account for the radiation that would otherwise be emitted by these galaxies we extract the LW emission from the original simulations and create a table of LW values that this target halo is exposed to as a function of
 redshift. Using the \grackle \citep{Grackle} software library we then use these LW tables as the ``background'' that this halo is exposed to. Self-shielding to LW radiation is also invoked in the re-simulations based on the \cite{Wolcott-Green_2011} prescription.\\
 \indent For the re-simulation of \ha the LW radiation background follows exactly
 the radiation field as determined from the original simulations and is shown
 in Figure \ref{Fig:LWHistory}. For \hb we purposely set the background radiation field to zero. This allows us to study the evolution of \hb with 
 star formation suppression due to dynamical heating only. \hb is therefore the 
 control simulation and allows us to examine the impact of what happens when
 no LW field is present. 
 

 \subsection{Subgrid Star Formation Prescription} \label{Sec:StarFormation}
 \indent In order to resolve star formation in the collapsing target haloes we set the maximum refinement
 level of the simulation to 20. This is an increase of a factor of $2^8 (256)$ compared to the
 original Renaissance simulations and allows us to reach a maximum spatial resolution of $\Delta x \sim 1000$~au.
 While this (maximum) resolution is less than what was achieved
 in \cite{Wise_2019} it was necessary as the goal of this re-simulation was not only to follow the
 collapse of the target halo but to also follow the formation of stars within the collapsing
 halo for up to 2 Myr following the formation of the first star. At the resolution used in
 \cite{Wise_2019} this proved intractable and so we reduced the
 resolution by a factor of $2^4 (16)$, compared to \cite{Wise_2019}, as a compromise. Reducing the
 refinement factor compared to \cite{Wise_2019} reduced the computational load while still allowing us to
 resolve star formation at an acceptable resolution. \\
 \indent In order to model star formation within the collapsing gas cloud we employed a star
 formation criteria using the methodology first described in \cite{Krumholz_2004}. The implementation
 in \enzo is described in detail in \cite{Regan_2018a} and \cite{Regan_2018b} and we give a
 brief overview here for completeness. Stars are formed when all of the following conditions are met:
\begin{enumerate}
\item The cell is at the highest refinement level
\item The cell exceeds the Jeans density 
\item The flow around the cell is converging 
\item The cooling time of the cell is less than the freefall time
\item The cell is at a local minimum of the gravitational potential
\end{enumerate}
Once the star is formed accretion onto the star is determined by evaluating the mass flux across a
sphere with a radius of 4 cells centered on the star. Initially all stars are assumed to be stars with low surface
temperatures that are appropriate for main sequence SMSs and less massive proto-stars on the
Hayashi track. The accretion onto the surface of the embryonic star is found by applying Gauss's
divergence theorem to the volume integral of the accretion zone \citep[e.g][]{Bleuler_2014}
(i.e. the volume integral of flux inside the accretion zone)
\begin{equation}
  \dot{M} = 4\pi \int_\Omega { \rho v_r^- r^2 dr}
\end{equation}
where $\dot{M}$ is the mass accretion rate, $\Omega$ is the accretion zone over which we integrate,
$\rho$ is the
density of the cells intersecting the surface, $v_r^-$ is the velocity of cells intersecting
the surface which have negative radial velocities and $r$ is the radius of our surface. \jhw{This is not a surface integral if $r$ is the integration variable. Here we state that $r$ is the radius of the surface, which is constant. The integration element should be $dA = r^2 d\Omega$. How is $d\Omega$ calculated in terms of the cell width?} \jr{I think this is ok. I am applying Gauss's divergence theorem to the volume integral (hence r enters) which is an easy way to compute the flux passing through the surface. I think I was missing a 4 $\pi$ though. That's corrected now and I've also clarified that I use Gauss's theorem}. As noted above we
set the accretion radius to be 4 cells. The accretion onto the star is calculated at each timestep,
however this is likely to be a very noisy metric. To alleviate this to some degree we average
the accretion rate over intervals of 1 kyr and use that averaged accretion rate in data outputs. The accretion rate is added as an attribute to each star and hence a full
accretion history of every star is outputted as part of every snapshot. Mergers with other stars
are also included in the accretion onto the stars. In this case the more massive star retains its
information (e.g., age, type, etc.) after the merger event - information on the less massive star is
lost. The mass of the less massive star is added to the accretion rate of the more massive star for
that timestep. Stars are merged when they come within 3 times the accretion radius of each other (i.e. 12 cell lengths).
\jhw{Where does this factor of three come from? Krumholz+ (2004)? In my SS-runs, I've set it equal to the accretion radius.}
\jr{It's just the code choice I made. It's not clear what the merger radius should be. Krumholz put theirs the
  same as the accretion radius but that's not required per-se. Setting it to be the accretion radius means that
  particles are very close before being merged. Twice the accretion radius means the accretion radii are touching.
  I felt three was ok too since below this scale we enter a regime where we possibly start to lose resolution.
  Three is probably the maximum in fairness. I tested this extensively at the time and found it worked well.}
  
\indent Each star also has the ability to provide both radiative and mechanical feedback, which is
most appropriate in the case where the star has transitioned into a (accreting) black hole. As the
accretion rate onto the star varies the star can transition its type from a SMS, with an inflated
surface, to a PopIII star.  This transitioning only occurs if the accretion rate onto the star either
never exceeds the critical rate of 0.04 \msolaryr or if the accretion rate onto the star falls below
the critical accretion rate \citep{Sakurai_2016}. While the star remains bloated the radiative
feedback from the star is primarily below the hydrogen ionisation limit and is mostly in the
form of infrared radiation. However, if the accretion rate drops and the star contracts to the
main sequence then its surface temperature dramatically increases up to $10^5$~K,
causing its spectrum to harden and peak in the UV. \\
\indent All stars in this simulation emit radiative feedback below the ionisation threshold of
hydrogen. The radiation is followed explicitly using the ray tracing technique \citep{WiseAbel_2011}.
As stars in this simulation contract to the main sequence, when the accretion rate onto the
star drops, radiation above the hydrogen ionisation threshold is not initiated. The shocks generated near
the star particle are too strong and unresolved for the PPM reconstruction scheme and HLLC Riemann solver
to handle, where the steep gradients caused the solution to overshoot to negative densities and
energies \citep[see][for details]{Enzo_2014}. We could have avoided this problem by utilizing a more
diffusive solver at the expense of accuracy, but instead
we neglect the impact of ionising feedback and post-process the impact of
ionising feedback with \cloudy and use the results to estimate the point at which the impact of
radiative feedback on the surrounding gas becomes so strong that the simulation should be terminated.
%%%%%%%%%%%%%%%%%%%%%%%%%%%%%%%%%%%%%FIGURE 3%%%%%%%%%%%%%%%%%%%%%%%%%%%%%%%%%%%%%%%%%%%%%%%%%%%%%%%%%%%
%%%%%%%%%%%Visualisations of haloes at start and end %%%%%%%%%%%%%%%%%
\begin{figure*} 
\centering
\begin{minipage}{175mm}      \begin{center} 
\centerline{
\includegraphics[width=0.52\textwidth]{FIGURES/HaloA/Proj_z_number_density_0001.pdf}
\includegraphics[width=0.52\textwidth]{FIGURES/HaloB/Proj_z_number_density_0028.pdf}}
\caption{Both panels show the projected number density of the region around which the first
  star forms. \ha is in the left hand panel, \hb in the right hand panel.
  The legend in each panel gives the mass of each star at the first output time following star
  formation as well as the age of each star at that time. The extent of each panel is 2 pc (physical). The orange
  star in each case represents the most massive star. Stars coloured in red are stars in which the accretion rate
  exceeds that required for supermassive star formation (0.04 \msolaryrc). Blue stars are those stars
  (normal PopIII stars) for
  which the rate is below the critical rate. \ha contains a single initial star. \hb contains two stars (closely separated)
  at the first output following star formation. For illustrative purposes the size of each star in the
projection is scaled as $\rm{R_{star}} \propto \rm{M_{star}^{0.6}}$.}\label{Fig:ProjectionStart}
\end{center} \end{minipage}

\end{figure*}
%%%%%%%%%%%%%%%%%%%%%%%%%%%%%%%%%%%%%%%%%%%%%%%%%%%%%%%%%%%%%%%%%%%%%%%%%%%%%%%%%%%%%%%%%%%%%%%%%%%%%%%%%

\subsection{Post-Processing with Cloudy}
\label{cloudy:description}
{\sc Cloudy} \citep{Ferland_2017} is a spectral synthesis code which models radiative transfer through a gas, and its resulting thermal and chemical equilibrium, under a wide range of conditions encompassing those expected for interstellar matter. %For primordial gas in the density and temperature range presently considered, 
The code relies on a number of databases for computing the behaviour of atoms and molecules, including tabulated recombination coefficients obtained from \cite{Badnell_2003} and \cite{Badnell_2006}, with Case A and B recombination predictions for single-electron systems from \cite{Storey_1995} and He I recombination rates from \cite{Porter_2012}, as well as ionic emission data from the CHIANTI database \citep{Dere_1997,Dere_2012}. Its chief limitation is that it is largely constrained to modelling environments in 1-D and/or assuming spherical symmetry \citep[though see, e.g.,][for recent efforts to extend its implementation to pseudo-3D problems]{Morisset_2013, Fitzgerald_2020}. For this reason, and because we do not follow the stellar evolution of each formed star in detail, we must make a number of simplifying approximations in using {\sc cloudy} to estimate the impact of ionising feedback. 

For each halo, we consider the ionising feedback at times 503~kyr, 732~kyr, and 998~kyr for \hac, and times 469~kyr, 732~kyr, 994~kyr, and 1354~kyr for \hbc. For each snapshot, we divide the stars formed between those which are accreting $> 5 \times 10^{-3}$ \msolaryr, which will evolve on the Hayashi track due to $\rm{H}^{-}$ opacity in their inflated envelopes \citep[e.g.,][]{Hosokawa_2013}, and those accreting below this threshold, which will evolve blue-ward as they contract to become hot ionising sources on the ZAMS \citep{Haemmerle_2017}. We assume that all rapidly-accreting (``red'') stars remain negligible ionising sources, and that all slowly- or non-accreting (``blue'') stars have thermally-relaxed to a main sequence temperature of $\approx 10^{5}$K \citep{Schaerer_2002, Woods_2020}. We further take their luminosities to be approximately Eddington (L $\approx 1.3\times 10^{38} \times (M/M_{\odot})$ erg s$^{-1}$), and their spectra to be well-approximated as blackbodies.

For each blue star, we then model the ionisation state of the surrounding gas assuming spherical symmetry, with the density profile found from the average gas density in successive shells 0.5pc in width centred on each star in the halo, and primordial abundance ratios taken to be 1.0:0.08232:1.6e-10:1e-16 for H:He:Li:Be \citep[consistent with the results of the][table 2, see {\sc cloudy} documentation for further discussion]{Planck_2014}. We assume an inner boundary of $10^{15}$ cm and terminate our calculations at the outer boundary of each nebula once either the gas temperature falls to 8000 K or the electron fraction falls below 5\%. Modelling the ionized nebula associated with each star in this way does not account for the overlapping of Str{\"o}mgren spheres associated with distinct stars; we address this point and further limitations in $\S$\ref{cloudy:results}.


\section{Results} \label{Sec:Results}

\subsection{Conditions for (Super-)Massive Star Formation}
As noted in \S \ref{Sec:Introduction} the goal of this study is to model the formation of SMS formation in haloes which are experiencing both LW feedback from nearby galaxies and dynamical heating from mergers. Both of these effects suppress the formation of regular PopIII stars in haloes
below the atomic cooling limit and hence may lead to the formation of a SMS in a larger halo  \citep[as was found in][]{Wise_2019}. In this study we include 
sub-grid star formation prescriptions to follow the star formation process in 
two of these candidate haloes.\\
\indent In the left hand panel of Figure \ref{Fig:LWHistory} we show the LW history that \ha is
exposed to throughout its re-simulation. As discussed in \S \ref{Sec:Methods} the LW field at the
location of the halo is determined from the original simulation, which included star formation
in all of the surrounding haloes. In the re-simulation the AMR grids are focused only around the
target halo and so the LW field must be imposed as a global background, albeit focused on a single
halo. As can be seen in the left hand panel of Figure \ref{Fig:LWHistory} the LW field is very flat and sits below \JLW \ $\sim 2$
J$_{21}$ until a redshift of $z \sim 16.5$ at which point it increases strongly due to a nearby source
which undergoes a starburst and emits copious amounts of LW radiation
(and ionising radiation\footnote{The
  shorter mean free path of the ionising radiation means that \ha does not get photoionised.}) in the
direction of \hac. The LW field reaches its zenith at a redshift of $z \sim 15.8$ and thereafter
starts to decline. The green circle at $z = 15.6$ represents the redshift at which the halo was
detected as a metal-free and star-free atomic cooling halo in the original simulations
\citep[see][for details]{Regan_2020}. The black star indicates the redshift at which star formation begins in this
re-simulation ($z = 15.05$). \\
\indent In the right hand panel we show the merger history of both \ha and \hb
as found in the original simulations. The mergers, both major and minor, drive the dynamical heating
which heats the gas and delays star formation \citep{Wise_2019}. 
\ha experiences a major merger starting at $z \sim 16.2$ and lasts until $z \sim 15.7$. This
merger, in combination with the LW field, suppresses star formation until z = 15.05 (again the green
circles denotes the time of collapse in the original, lower resolution, simulations). The merger
history of \hb is shown as the red line. Recall that \hb is run without any LW background as a control case. As a
result star formation is triggered at $z = 17.21$. Using the results from the original simulation
which allows us to see the future evolution of that halo we see that immediately after star formation
the halo undergoes a major merger. It is therefore highly likely that, in the absence of a suppressing
LW background, the merger triggers star formation, which in the original simulation was
suppressed until after $z = 15$. Therefore, we can see here that the absence of the background
LW radiation field was sufficient to allow PopIII star formation to take place. Dynamical heating
by itself was not sufficient for this halo to avoid star formation. \\
\indent Star formation occurs in both \ha and \hb when they are at very different phases in their evolution.
In \ha the mass of the halo is significantly above the atomic cooling limit
(M$_{\rm{HaloA}} = 9.3 \times 10^7$ \msolarc) and the halo has
remained metal-free. \hbc, on the other hand, collapses early in a
halo with a mass of M$_{\rm{HaloB}} = 3.7 \times 10^6$ \msolar due to the lack of LW background and thus can
be classified as a mini-halo. Before analysing the star formation in detail we use results from \cloudy to determine when the negative feedback from stars eventually
quenches further star formation through photoionisation in each halo. 
%%%%%%%%%%%%%%%%%%%%%%%%%%%%%%%%%%%%%FIGURE 4%%%%%%%%%%%%%%%%%%%%%%%%%%%%%%%%%%%%%%%%%%%%%%%%%%%%%%%%%%%
%%%%%%%%%%%Visualisations of haloes at start and end %%%%%%%%%%%%%%%%%
\begin{figure*}
\centering
\begin{minipage}{175mm}      \begin{center} 
\centerline{
\includegraphics[width=0.52\textwidth]{FIGURES/HaloA/Proj_z_number_density_2001.pdf}
\includegraphics[width=0.52\textwidth]{FIGURES/HaloB/Proj_z_number_density_1103.pdf}}
\caption{Both panels show the projected number density in each halo at the end of each simulation. The
  left panel shows \ha while the right panel shows \hb.
  The legend in each figure gives the mass of the five most massive stars at the final output time, as well as
  the age of the star at that time. The extent of each panel is 20 pc (physical). The orange
  star in each case represents the most massive star. Blue stars are those stars (normal PopIII stars) for
  which the rate is below the critical rate (0.04 \msolaryrc) for SMS formation. At the times shown no
  SMSs exist in either simulation because the accretion rate onto each star is less than the
  critical rate.  For illustrative purposes the size of each star in the
  projection is scaled as $\rm{R_{star}} \propto \rm{M_{star}^{0.6}}$.}  \label{Fig:ProjectionEnd}
\end{center} \end{minipage}

\end{figure*}
%%%%%%%%%%%%%%%%%%%%%%%%%%%%%%%%%%%%%%%%%%%%%%%%%%%%%%%%%%%%%%%%%%%%%%%%%%%%%%%%%%%%%%%%%%%%%%%%%%%%
%\begin{table*}
    %\centering
    %\begin{tabular}{c|ccccc}
         %Halo A &  Time [kyr] & \% ion. volume ($<1$pc) & \% ion. volume %($<5$pc) & ion./rec. ($<1$pc) & ion./rec. ($<5$pc)\\
%         \hline
%          & 503 & 8.3 & 0.2 & & \\
%          & 732 & 100 & 3.2 & & \\
%          & 998 & 2.0 & 27 & & \\
%          Halo B & & & & & \\
%          \hline
%          & 469 & 0.13 & 2e-3 & & \\
%          & 732 & 14 & 0.7 & & \\
%          & 994 & 100 & 2.2 & & \\
%          & 1354 & 100 & 3.9 & & \\
%    \end{tabular}
%    \caption{Caption}
%    \label{tab:my_label}
%\end{table*}

%% \begin{table}
%%     \centering
%%     \begin{tabular}{c|ccc}
%%          Halo A &  Time [kyr] & \% ion. volume ($<5$pc) & ion./rec. ($<5$pc)\\
%%          \hline
%%          &  1  & 4.9e-7 & 1.7e-4 \\
%%          & 209 & 2.0e-7 & 4.0e-5\\
%%          & 318 & 2.3e-6 & 3.2e-5\\
%%          & 427 & 8.6e-5 & 1.2e-5\\
%%           & 503 & 0.23 & 6.2e-6\\ %
%%           & 732 & 3.1 & 6.4e-7\\ %
%%           & 998 & 18 & 4.3e-5\\ %
%%           & 1107 & 21 & 7.3e-5\\
%%           & 1282 & 33 & 2.8e-5\\ %
%%           & 1412 & 39 & 6.9e-6\\ %
%%           & 1500 & 40 & 9.4e-6\\%
%%           & 1609 & 44 & 3.6e-6 \\ %
%%           & 1718 & 39 & 1.0e-5 \\ %
%%           & 1783 & 36 & 3.3e-6 \\ %
%%           & 1892 & 46 & 5.7e-6 \\ %
%%           & 2023 & 46 & 5.5e-6 \\ %
%%           Halo B &  Time [kyr] & \% ion. volume ($<1$pc) & ion./rec. ($<1$pc)\\
%%           \hline
%%           & 28 & 1.3e-4 & 1.3e-4\\
%%           & 109 & 4.0e-4 & 1.3e-4\\
%%           & 200 & 4.3e-5 & 6.3e-5\\
%%           & 277 & 1.6e-4 & 1.6e-4\\
%%           & 381 & 4.2e-5 & 1.3e-4\\
%%           & 469 & 0.14 & 2.9e-5\\
%%           & 732 & 12 & 1.7e-3\\
%%           & 994 & 70 & 0.52\\
%%           & 1103 & 93 & 6.1\\
%%           & 1354 & 99 & 63\\
%%     \end{tabular}
%%     \caption{For a selection of snapshots in the evolution of \ha and \hb we present the fraction of the inner star-forming volume (as a percentage) which is expected to be ionised, as well as the total ionising photon luminosity of the embedded stars as a fraction of that required to wholly ionise the gas within the same volume. These two measures are distinct but complementary.}% Due to significant density variations and our density-averaging procedure.
%%     %(as well as our neglect of overlaps in computing the volumes of Str{\"o}mgren spheres) 
%%     %these two measures are distinct but complementary.}% $^{a}$Note that at this snapshot, the total ionized volume within 1pc of the largest star is driven by two objects: the largest star itself ($\approx 173M_{\odot}$) and a neighbouring $\approx 110 M_{\odot}$ star approximately 0.5pc away, both with Str{\" o}mgren radii of approximately 0.8pc; the neglected overlap of these two regions largely accounts for what is clearly an overestimate of the ionized volume, as evident from the insufficient ionizing photon budget to account for all of the matter within 1pc.}
%%     \label{tab:feedback}
%% \end{table}

\subsection{Determining the time of star formation quenching}\label{cloudy:results}

To assess the impact of photoionising stellar feedback, we must investigate the ionisation state
of the gas in the innermost star-forming regions of each halo. Here we adopt an heuristic
definition of ``innermost region,'' based on their relative sizes and masses and the star
formation seen in our models. For \ha we adopt an inner region of 5 pc. Using the most massive
star as the centre approximately 90\% of the stellar mass resides inside this inner region. For \hb
we adopt an inner region of radius 1 pc centred on the most massive star. For \hb 90\% of the
stellar mass resides inside this inner 1 pc region.   
%for \ha we adopt the inner 5 pc, and for \hb the inner 1 pc, in each case centred on the most massive star within each halo at each chosen time (for \ha 503, 732, 998, 1107, 1282, 1412, 1500, 1609, 1718, 1783, 1892, and 2023 kyr; for \hb 469, 732, 994, 1103, and 1354 kyr).
\jhw{Why did you choose 5pc and 1pc? Is there a physical scale that we can relate to, like the Jeans mass?} \jr{It was an estimate based on the separation of the stars. This separation tends to
  increase by a factor of a few though over time. 5 pc and 1 pc are very good for the first
  few hundred kyr but after say a Myr the stars do spread out. I've asked Tyrone to redo the
  cloudy analysis using a volume of 20 pc and 5 pc respectively to see how it changes the result}
We may then evaluate the evolution of both the total ionised volume and the total ionisation budget through each snapshot, in order to provide subtly distinct but complementary measures of the strength of stellar feedback.

First, we compute the ionised volume associated with each thermally-relaxed star as described in \S\ref{cloudy:description}, and integrate the intersection of the union of these Str{\"o}mgren regions with the central star-forming region, as defined in the preceding paragraph for each halo, using a Monte Carlo approach.
%For simplicity, here we neglect any overlap between the Str{\" o}mgren regions associated with each star, although we note a number of such overlapping regions at late times. A more precise estimate of the ionised volume would however necessitate a more detailed 3-D treatment of the radiative transfer which, as discussed above, is beyond the scope of this work; furthermore, 
Note that for a constant combined ionizing luminosity, any reduction in ionised volume in accounting for overlaps in our spherically-symmetric Str{\"o}mgren spheres would in principle be compensated by the ionisation of a greater (though presumably not spherically symmetric) volume by the remaining available ionising photons. The extent to which this would raise the total ionized volume beyond our estimate depends sensitively on the local density distribution beyond each Str{\" o}mgren region.

Indeed, an alternative approach is to simply compare the ionizing photon luminosity of all stars within the inner star-forming region with the total integrated recombination rate of the gas if all hydrogen atoms were ionized; the latter sets the total emission rate of photons with E$>$13.6eV needed to maintain ionization of the star-forming region in equilibrium. This can be found from integrating over the hydrogen number density:

\begin{equation}
    \dot R = \int_{\rm{V}}\alpha_{\rm{B}}(\rm{H}^{0},T)n_{\rm{H}}^{2}(r)d\rm{V}
\end{equation}

\noindent where $n_{\rm{H}}(r, t)$ is the spherically-averaged density profile centered on the most massive star in the Halo at time $t$ (see above), and $\alpha_{\rm{B}}(\rm{H}^{0},T)$ is the case B recombination coefficient for hydrogen, which we take as $\approx1.4\times 10^{-13}\rm{cm}^{3}\rm{s}^{-1}$ for primordial gas at a temperature typical of plasmas photoionized by Population III stars, $\rm{T}\sim 2\times 10^{4}$K \citep{Osterbrock2006, Johnson2012}. 

These two measures of the strength of ionizing stellar feedback are compared in Figure \ref{Fig:Ionisation}. The left panel shows the volume ionised fraction of the innermost region as a function of time, while the right panel shows
the fractional ionising budget as a function of time.  \jhw{It might also be interesting to show the average Str{\"o}mgren radius, i.e. $(3V/4\pi N_\star)^{1/3}$ in the left panel.} Note that the ionized volume (left panel) and the fractional ionising photon budget (right panel) do not necessarily
move perfectly in tandem. This is partly due to our simplified treatment as described above, including overlapping Str{\" o}mgren regions and spherically-averaged density profiles, but also
reflects the variable density distribution, with the lowest density regions preferentially ionised over high-density knots. In particular, in both \ha and \hb we see a marked rise in the
ionized volume between $\sim 500$ and $\sim1200$ kyr, however only for \hb is this matched by a similar rise in the fractional ionizing photon budget. For \hb both measures show a fully ionised medium
after approximately 1000 kyr. \ha on the otherhand shows an initial ionisation of the central gas medium but that ionisation level saturates at approximately 40\% after approximately 1500 kyr. The fractional
ionisation budget remains very low with recombinations dominating - this relfects the web of dense filaments and knots that pervade the inner regions of \hac. \\
\indent In summary, ionizing stellar emission is likely to shut down star formation in \hb $\sim 1.1$ Myr after the first stars have formed. At the same time in its evolution, the much denser gas in the central region of \ha is more robust against stellar feedback, albeit with the latter growing in importance at this point. Our results indicate that star formation and accretion in \ha are unlikely to be shut down prior to the onset of supernova feedback and chemical enrichment of the halo, before $\sim$ 2 Myr, at which time the first generation of stars reaches the end of their lives. 

 
%%%%%%%%%%%%%%%%%%%%%%%%%%%FIGURE 5%%%%%%%%%%%%%%%%%%%%%%%%%%%%%%%%%%%%%%%%%%%%%%
%Mass Function
\begin{figure*}
\centering
\begin{minipage}{175mm}      \begin{center}
\centerline{
    \includegraphics[width=18.0cm, height=12cm]{FIGURES/FinalMass_MultipleHaloes.pdf}}
\caption{
  The mass function for the stars that formed in \ha over 2 Myr and \hb over 1.1 Myr.
  Blue bars represent stellar masses from \ha while red bars represent stellar masses from \hb.
  The dichotomy in masses in clearly evident with a strong bias towards more massive stars
  in \hac. This is due to the higher temperatures in \ha compared to \hb which in turn
  leads to high infall rates to the centre and hence more mass available for proto-stars
  to accrete. The median mass for a single star in \ha is 680 \msolar and the median mass
  for \hb is 44 \msolarc. 
}
\label{Fig:MassFunction}
\end{center} \end{minipage}
\end{figure*}
%%%%%%%%%%%%%%%%%%%%%%%%%%%%%%%%%%%%%%%%%%%%%%%%%%%%%%%%%%%%%%%%%%%%%%%%%%%%%%%%%

%%%%%%%%%%%%%%%%%%%%%%%%%%%%%%FIGURE 6%%%%%%%%%%%%%%%%%%%%%%%%%%%%%%%%%%%%%%%%%%%%%%%%
% Radial Profiles prior to formation
\begin{figure*}
\centering
\begin{minipage}{175mm}      \begin{center}
\centerline{
    \includegraphics[width=18.0cm, height=12cm]{FIGURES/MultiPlot.pdf}}
\caption{We examine the radial profiles for both \ha (blue line) and \hb (red line). The four panels
  are from bottom left going clockwise: temperature, \molH fraction, enclosed gas mass and infall rate.
  As \hb collapsed early due to the lack of a LW background (compared to the original Renaissance simulation)
  the temperature of this halo is significantly lower in the centre. The central temperature equilibrium
  value is close to 500 K which is characteristic of PopIII simulations for the chemical network used in this work.
  This lower temperature is due to the high \molH fraction, as shown in the upper left panel. This can
  be contrasted with the radial profiles for \ha which show systematically higher temperatures in the core of the
  halo and lower \molH fractions. The lower \molH fractions are due to a combination of dynamical heating from
  mergers and from local radiation sources. The enclosed mass fractions identify the difference
  in masses between the haloes. In the case of \hb the enclosed masses are systematically lower compared
  to \ha as this is a effectively a minihalo. Infall rates in both haloes are very different in the
  centre of each halo. Average infall rates in the very centre (R $\lesssim 0.1$ pc) of \ha are
  above 0.1 \msolaryr compared to below 0.001 \msolaryr in \hbc. These infall rates are consistent
  with the accretion rates that are seen onto the stars that subsequently form in each halo. The flat
  patches in the infall rate are due to outflow at those radii
  (which show up as negative infall rates). }
\label{Fig:RadialProfiles}
\end{center} \end{minipage}
\end{figure*}
%%%%%%%%%%%%%%%%%%%%%%%%%%%%%%%%%%%%%%%%%%%%%%%%%%%%%%%%%%%%%%%%%%%%%%%%%%%%%%%%%%


%%%%%%%%%%%%%%%%%%%%%%%%%%%%%FIGURE 6%%%%%%%%%%%%%%%%%%%%%%%%%%%%%%%%%%%%%%%%%%%%%%%
% Accretion Rates and Masses for all haloes. Most massive and median

\begin{figure*}
\centering
\begin{minipage}{175mm}      \begin{center}
    \centerline{
      \includegraphics[width=9.0cm, height=6cm]{FIGURES/Mass_MultipleHaloes.pdf}
      \includegraphics[width=9.0cm, height=6cm]{FIGURES/MassAccretionRate_MultipleHaloes.pdf}}
    \caption{
      \textit{Left Panel:} The mass evolution of the most massive star in each simulation. The
      solid blue line is for the most massive star in \texttt{HaloA}. It has a mass of
      6128 \msolar after 143 kyr. The solid red line is for the most massive star in \texttt{HaloB}.
      It has a mass of 173 \msolar after 1162 kyr of evolution. We also look at the mass evolution
      of an average star in each simulation for comparison. In \ha median$^4$ star has a mass of 683
      \msolarc, this is shown by the dashed blue line. In \hb on the other hand a median star has a
      mass of 43 \msolar and is shown by the dashed red line. 
      \textit{Right Panel:} The accretion rate onto the most massive star for both \ha and \hb
      (solid lines), dashed lines give the accretion rate onto the median mass star for both \ha
      and \hbc. In both cases initial accretion rates begin to decline rapidly after 10 -- 20 kyr and
      do not recover. The black dashed line gives the critical rate for SMS star formation
      \citep{Sakurai_2016}.
    }
\label{Fig:AccretionRates}
\end{center} \end{minipage}
\end{figure*}

%%%%%%%%%%%%%%%%%%%%%%%%%%%%%%%%%%%%%%%%%%%%%%%%%%%%%%%%%%%%%%%%%%%%%%%%%%%%%%%%%%%%%%%%%



\subsection{The onset and end of star formation}
Star formation is triggered when all of the criteria set out in \S \ref{Sec:StarFormation} are
fulfilled. In Figure \ref{Fig:ProjectionStart} we show a projection of the total gas number density\footnote{Number density refers here to the number density of each separate species weighted by each species atomic mass unit.} in the
regions surrounding the formation of the first star in each halo. The projections are made from the
first snapshot following star formation. Each panel is 20 pc across. In the left hand panel we
show the projection from \ha, which shows the formation of a single star (coloured in orange
at the centre of the green coloured gas cloud southwest of centre). The legend on the top left gives the mass and
age of the star at this time.
%Stars coloured in orange denote the most massive star in the simulation.
In the right-hand panel we show the projection for \hbc. In this case two stars are formed by the time the first
output following star formation. The most massive star has a mass of 37 \msolar and is 28 kyr old,
the second star has a mass of 6 \msolar and is 4 kyr old. Stars are coloured in red if the
accretion rate exceeds the SMS critical rate of 0.04 \msolaryrc, otherwise 
stars are coloured in blue (denoting PopIII stars). The most massive star is
coloured in orange. \\
\indent In Figure \ref{Fig:ProjectionEnd} we show the extent of star formation by the end of each
simulation (as determined by the \cloudy models). In the left panel we show the results for \ha and
for \hb in the right hand panel. Simulation \ha is terminated approximately 2 Myr 
after the formation of the first
star (due to the imminent impact of chemical and mechanical feedback from the first supernovae) while \hb is terminated approximately
1.1 Myr after the formation of the first star (due to the photoionisation feedback predicted by the \cloudy models). At the end of the \ha
simulation there are 99 stars in the simulation with masses ranging from approximately 40 \msolar
up to a maximum mass of over 6000 \msolarc.  Star formation in \ha is widespread throughout the
inner 20 pc of the halo with a number of different gas clouds giving rise to star formation. Furthermore, as the simulation develops the interactions between clouds triggers star formation as individual clouds
merge and interact. \hb, on the other hand, contains essentially only a single site of star formation
due to the significantly smaller halo mass. At the end of the simulation of \hb there are 21 stars
with masses ranging from approximately 20 \msolar up to approximately 170 \msolar.\\
\indent In Figure \ref{Fig:MassFunction} we show the mass distribution of stars at the final
output time. Stars from \ha are binned in blue, stars from \hb are binned in red. The median mass
of stars formed in \hb is 44 \msolar while the median mass of stars formed in \ha is 680 \msolarc.
The most massive star in \ha at the final output time is 6128 \msolar with another star with a mass
of 4477 \msolarc. \ha has 22 stars with masses exceeding 1000 \msolarc. The lowest mass star in
\ha has a mass of 40 \msolarc. The mass distribution of stars in \hb is significantly smaller, running
from 22 \msolar up to 173 \msolarc. However, it should be noted that due to our finite resolution
($\Delta x \sim 1000$ au) we cannot put a lower limit on star formation and we are likely missing lower mass stars. 

\subsection{The case for super-massive star formation}
In Figure \ref{Fig:ProjectionEnd} we saw that the mass of the most massive star in \ha is
6128 \msolarc. While this mass is well beyond the mass of ordinary PopIII stars \citep{Turk_2009, Greif_2011, Wise_2012b, Crosby_2013, Susa_2014, Hirano_2014, Stacy_2016} and also more than 30 times
larger than the most massive star in \hb it is still well short of the mass often associated with
truly super-massive stars \citep[e.g.][]{Woods_2018} which are expected to have
end stage masses of $\sim 10^5$ \msolar. Furthermore, \ha was chosen here to present near ideal initial conditions in which to form a supermassive star.
To illustrate this point we show radial profiles of the halo properties of both \ha and \hb in
Figure \ref{Fig:RadialProfiles}. Figure \ref{Fig:RadialProfiles} shows the state of the gas in each halo immediately prior to star formation. \ha is denoted by the blue line, \hb by the red line.
The temperature, \molH fraction, enclosed gas mass and mass inflow rate are shown in clockwise order starting from the bottom left panel. \\
\indent \ha shows near isothermal collapse, although the gas does
show some degree of cooling at approximately 10 pc. The \molH fraction is very different between
the two haloes, with \ha showing a steep decline in \molH towards the centre due to the combination
of the LW background and the rapid assembly of mass. \jhw{I don't think it's caused by the LW background because the core would be shielded and have a higher H2 fraction than the outer halo.  But I agree that the dissociation could be caused by the rapid assembly, higher infall rates, and higher post-shock temperatures. But on the other hand, the increased electron fraction in the stronger shocks should promote H2 formation.} \jr{The self-shielding kicks in at about 10 pc though and the decline in temperature
  occurs outside of that radii. Would you not think that some of the \molH decline could be attributed to the LW radiation? I think some of the heating is due to the infall and the associated KE of the gas and no coolant being available. }\hb on the other-hand shows a more typical
\molH evolution consistent with mini-halo formation. The enclosed mass plot in the top right panel
illustrates the different mass associated with each halo as a function of radius. \ha being well
inside the atomic cooling mass range. In the bottom right hand panel we see the mass inflow rate for
each halo. \ha shows mass infall rates averaging greater than 0.1 \msolaryr all the way into the
centre of the halo. \hbc's mass infall rates on the other hand fall from approximately
0.1 \msolaryr at 1 pc down to approximately $10^{-3}$ \msolaryr in the very centre. As we will see
this result is reflected in the accretion rates observed onto the protostars. \\
\indent In order to understand why, given apparent high mass infall rates that are greater than the
critical rate, a SMS does not form we need to examine the accretion rate that is measured onto the
stars within the simulation itself. In Figure \ref{Fig:AccretionRates} we show the mass evolution and
the accretion rate onto the most massive star in each halo. In the left panel we show the mass
evolution as a function of the stellar age. The most massive star in \ha is denoted by the blue line,
the most massive star in \hb by the red line. The dashed line in the panel shows the mass evolution
of a star with the median\footnote{We define a
        median star here as the star at the end of the simulation which has a final mass
        closest to the median stellar mass at the end} stellar mass in the simulation. The most massive star in \ha is formed
when two gas clouds within the central region merge and trigger star formation. The star quickly
grows in mass up to its final mass of 6128 \msolar within approximately 20 kyr after which
it stops growing. The mass evolution of the most massive star in
\hb is a little different. In this case the star accretes slower at approximately $10^{-3}$
\msolaryr (see right panel). What then drives the star to increase its mass is a stellar merger
with another star. This occurs when the main progenitor star is approximately 200 kyr old and
results in a star of 173 \msolarc. \jhw{Is this physical, especially since it's 200~kyr after its creation?  We might want a statement giving a caveat or two.} \jr{I think 200 kyr is ok. That's a long time after its creation. That's thousands of timesteps?}\\
\indent In the right hand panel of Figure \ref{Fig:AccretionRates} we see the accretion rates onto
the most massive star and onto the median star for each simulation. Initially each star follows approximately the mass accretion rate as found in the radial profiles (see Figure \ref{Fig:RadialProfiles}).
However, in both cases after a few tens of kyrs the accretion rate onto each star drops dramatically and mass growth is halted. This drop in accretion is not due to feedback since no ionising feedback
is present in these simulation. Instead mass accretion is terminated as the star accretes all the gas in its surroundings and moves into a less dense region of the inner halo. All of the stars formed have a small initial velocity of a few \kms, which it takes from the gas that initially formed the star. While this 
initial velocity is much smaller than the circular
velocity of the host halo it does cause the stars to move around and decouple from the high density gas cloud in which they initially formed. \\
\indent By the end of each simulation the total stellar mass in \ha is 90,000 \msolar and 1300
\msolar in \hbc. The star formation efficiency in the inner 20 pc (5 pc) of \ha (\hbc) is 28\%
(18\%). 20 pc is the approximate Jeans length of the gas in \ha while 5 pc is the approximate Jeans
length of the gas in \hbc. In order for the formation of a truly supermassive star in \ha all of the
stellar mass
(i.e. $\sim$90,000 \msolarc) would have needed to have been accreted onto the stellar surface. Instead
what happened was that the outer gas cloud in \ha fragmented into a small number of sub-clouds which
each generated hotspots of star formation. These sub-clouds tidally disrupt each other during the
subsequent evolution both triggering further star formation but also tidally disrupting accretion.
From a total baryonic reservoir of approximately 300,000 \msolar inside 20 pc, approximately 2\% of the baryons went into the most massive star. To form a SMS
with a mass of close to 100,000 \msolar we would need a third of the mass to 
flow into a single object. While this is possible it is clearly going to be 
very challenging given the turbulent nature of the environment in which these stars are forming. What appears more likely, for this halo at least, is that the 
most massive star(s) in the halo will form of population of intermediate mass black holes (IMBHs) with masses ranging from 1000 \msolar to several thousand \msolarc. The subsequent growth of these IMBHs within the halo then through 
mergers and accretion will then slowly build the black hole mass. 
Nonetheless, the key issue with forming a monolithic SMS appears to be that
these haloes are highly turbulent and that sustaining accretion onto a single
object in such an environment is hugely challenging.
This conclusion is unlikely to be due to insufficient resolution: it is
well-known that low resolution damps turbulent motion very significantly
\citep[e.g.][]{Federrath_2010a, Downes_2012}. Thus insufficient resolution
will lead to an {\it over-estimate} of the likelihood of forming an SMS.

\section{Intermediate Mass Black Hole Merging and Implications for Gravitational Wave detection} \label{Sec:GW}
\note{I think we should analyze and give some dynamical statistics of the stars. Some suggestions are the distribution of their radii, eccentricies, orbital velocities, half-mass radii, stellar enclosed mass radial profile, and/or their rms velocity.  This would give some quantitative substance to this section and connection to the dense stellar cluster formation scenario.}

At the end of the simulation \ha has 22 stars with masses greater than 1000 \msolar with the most
massive star having a mass of 6128 \msolarc. The simulations show that accretion onto the stars is
focused on the early stages in the life of the star with most of the mass accreted within the
first 10 -- 20 kyr. Therefore, if we assume that the proto-galaxy gets populated with a plethora of
weakly accreting black holes early in the proto-galactic evolution then it is reasonable to ask \textit{what are the prospects for the merger of this first generation
  of black holes into a population of more massive black holes?}.\\
\indent \cite{Pfister_2019} investigated the merging of black holes in high resolution
simulations ($\Delta x \sim 10$ pc) with black hole masses in the range
$10^4$ \msolar to $10^5$ \msolar in high-z galaxies. They found that dynamical friction from stars
has a more stabilizing effect on the black holes than the gas component but that the stellar
distribution in high-z galaxies is highly irregular and hence is not able to stabilise the black
hole orbits. \jhw{Does the gas dynamical friction depend on the densities? I'd think that given our higher resolution and densities, the dynamical friction should be stronger.}  \jr{Yes the drag depending linearly on the density and the mass of the object (e.g. black hole). While our densities are higher our particle masses are also lower so on balance we may not see much difference?}
The irregular stellar distributions found by \cite{Pfister_2019} correlates with what
we see in our simulations here. \cite{Pfister_2019} found that black holes with masses less than
$10^5$ \msolar do not sink to the centre of a galaxy, but instead exhibit random walk characteristics.
However, their simulation neglected the impact of any radiative feedback from the black hole.\\
\indent \cite{Toyouchi_2020} investigated a more idealised scenario where they modelled the
dynamics of a single $10^4$ \msolar mass black hole incorporating the effects of dynamical friction
and radiative feedback. They found that if the gas density is low that the black hole
does not sink to the centre as found by \cite{Pfister_2019}. However, in the case where
the black hole encounters dense gas the ram pressure of the head wind (due to radiative feedback)
causes the black hole to lose orbital energy and so the black hole can sink to the centre on a
timescale much shorter than the dynamical time. Taking the two results together this
appears to imply that the specific environment that the black hole finds
itself in will play a central role in determining the timescale for the black hole
to sink towards the centre. If the black hole can interact with sufficiently dense gas it may
sink towards the centre (even for black holes with masses of $\sim 10^4$ \msolarc) while in
the case where the black hole does not encounter sufficiently dense gas then the black hole
may wander the central region of the galaxy for many dynamical times. The galactic centres
in which our massive PopIII
stars form (especially \hac) contain a web of dense gaseous filaments which may help to supply
the necessary ram pressure and extract orbital energy. 
What then does this mean for potential gravitational wave detection?\\
\indent Within approximately 4 Myr after the formation of the first massive stars here there
will be a population of black hole seeds with masses in the range 1000 \msolar up to 6000 \msolar
(and likely higher). In Figure \ref{Fig:Sensitivity} we plot the signal expected from the
merger of two black holes with masses of 5000 \msolar each at $z = 15$. These masses
are consistent with the masses of the stars found in this simulation and black holes
with similar masses are expected from the direct collapse of these massive PopIII
stars \citep{Heger_2003}. The merger of two black holes with these masses will produce a
gravitational wave signal detectable by LISA \citep{eLISA, Sesana_2016, Cornish_2020} emitting
gravitational waves between approximately $10^{-4}$ and $10^{-1}$ Hz (Detector frame frequencies). The
Signal-to-Noise ratio (SNR) from the merger of two black holes with masses of 5000 \msolar at $z = 15$
is approximately 33 and therefore should be detectable by LISA. The merger of a 10,000 \msolar binary
black hole system will enter the LISA band approximately 1 month before merger and complete
thousands of orbits before merger. With this number of cycles LISA will be able to detect
the redshifted mass with a precision of close to 1\% \citep{Sesana_2013}. The largest uncertainty
will however come from the determination of the redshift (or luminosity distance) which at this
redshift is expected to have an uncertainty of approximately 30\% \citep{Sesana_2013}. Additional
uncertainties due to weak lensing will also contribute at the percent level
\citep{Shapiro_2010, Petiteau_2011}
but the dominant error for these high-z sources will be LISA intrinsic errors on the measurement of
the wave amplitude. \\
\indent This redshift uncertainty will translate to a similar uncertainty in the chirp mass of the binary.
\cite{Klein_2016} modelled the merger of similar black hole masses and redshifts and using a
SNR = 7 found that the expected number of detections would be 358 over a 5 year mission period
(their \texttt{N2A5M5L6} model in Table II). Their modelling only accounted for the merging of
heavy seed black holes following a galaxy merger and
neglected the in-situ models as described here which could increase the numbers of events potentially
discoverable by LISA significantly. This point was previously investigated by
\cite{Hartwig_2018} as a method to discriminate between seed formation scenarios. Further,
investigation of the number density of in-situ mergers and how they could influence the number
of detections by LISA is now underway (Regan et al. in prep).




\section{Discussion and Conclusions} \label{Sec:Discussion}
The aim of this study was to examine the prospect of (super-)massive star formation in haloes
that were exposed to moderate LW radiation and also experienced significant dynamical heating
effects due to mergers. We identified 79 such haloes in the Renaissance 
datasets \citep{Regan_2020} and then subsequently selected two haloes, which showed no star formation or metal enrichment prior to reaching the
atomic cooling limit, for re-simulation at enhanced resolution. These halo characteristics make these environments ideal for forming massive
stars. \\
\indent The two haloes, \ha and \hbc, were then targeted for re-simulation by 
increasing the resolution and by adding additional physics capabilities to provide a more self-consistent calculation of (super-)massive star formation in these haloes. Firstly, the spatial resolution was increased by a factor of 16 ($\Delta x \sim 1000$ au) and the mass resolution by a factor of
169 (M$_{\rm DM} \sim 170$ \msolarc). Additionally, self-shielding of \molH from LW was included and the Jeans length was refined by at least 64 cells at all times.  To model star formation within the collapsing gas clouds a star formation prescription was employed, which additionally tracks accretion
onto the stars and allows for the distinction between compact PopIII stars and massively inflated
SMSs accreting above the critical rate of 0.04 \msolaryrc \ \citep{Sakurai_2016}.\\
\indent \ha was simulated with a LW radiation field impinging on it that followed exactly what was
found in the original simulations and so accounted for star formation from nearby galaxies. The
LW radiation field imposed is shown in Figure \ref{Fig:LWHistory}. \hb on the other hand was used as the
control halo and no LW field was imposed to test the impact that turning off the LW field would have.\\
\indent In the high-resolution re-simulation \ha formed 22 stars with masses greater than 1000
\msolarc. The most massive star in \ha has a mass of 6128 \msolarc. \hbc, re-simulated without
a LW radiation field, collapsed earlier in its evolution before reaching the atomic cooling limit.
The lack of any LW field allowing PopIII star formation to take hold. In \hb a total of 21 stars
formed at the end of the simulation with the most massive star in \hb having a mass of 173 \msolarc.  \jhw{Given that a few groups (e.g. Susa+, Hirano+, Hosokawa+) have looked at Pop III IMF in minihalos, we should compare with them.}
While the most massive star in \ha has a mass of more than 6000 \msolar its accretion rate had fallen
to less than $10^{-6}$ \msolaryr by the time the simulation was terminated. No ionising radiative
feedback was employed in these simulations and hence radiative feedback is not the reason for
the falloff in accretion. Instead, while high accretion rates of greater than 0.1 \msolaryr initially
fall onto the protostar those rates are not sustained. The turbulent nature of the collapsing parent
cloud causes a multitude of sub-clouds to form and dissipate. The sub-clouds with masses of
thousands of solar masses form stars but also tidally disrupt other sub-clouds disrupting
accretion onto other (proto-)stars. This phenomenon is observed in all of the stars formed in the simulation.
On average stars accrete above 0.01 \msolaryr for approximately 10 kyr before accretion is halted
due to external tidal disruption. \\
\indent What does this mean for the formation of SMSs and massive black hole seeds? This is only a
single halo simulated for 2 Myr. However, it is likely that the phenomenon of a turbulent environment is common in the early stages of galaxy formation and PopIII studies have also converged on the
formation of multiple stars in mini-haloes \citep[e.g.][]{Turk_2012}. It is therefore not surprising that this picture appears in
more massive galaxies as well. After approximately 2 Myr the stars in \ha will start to directly
collapse in black holes retaining close to 100\% of the stellar mass \citep{Heger_2003}. While this
event is unlikely to have any immediate impact on the mass accretion rate onto the black hole it will
leave this halo with a small population of massive black holes only a few Myrs after the onset of galaxy formation. Allowing the galaxy to evolve further  (through a few tens of dynamical times) may
reveal than these intermediate mass black holes can sink towards the centre of the halo and subsequently merge to form more massive black holes with masses close to $10^5$ \msolar (similar to what most canonical galaxy formation
simulation use as their seed mass). Such black hole mergers may even be
detectable by LISA as we demonstrate in Figure \ref{Fig:Sensitivity}. However, it is likely that some of this population of IMBHs will not merge or accrete any significant amount of gas, will display very low duty cycles and will wander their parent galaxy \citep{Tremmel_2018, Reines_2020, Barausse_2020}.\\
\indent To develop this model further what will actually be required is further high
resolution simulations which can model the formation of massive stars in high-infall rate haloes,
perhaps independent of metallicity \citep{Chon_2020, Regan_2020a}, which experience
a large transfer of baryonic mass towards their centres. These simulations will equally need to
be able to differentiate between PopIII star formation and SMS star formation, as done here, but
for a longer time period and beyond the formation of the subsequent black holes.
Tracking the evolution of these truly seed black holes is the next frontier. 

%%%%%%%%%%%%%%%%%%%%%%%%%%%%%%%%%%%%%FIGURE 7%%%%%%%%%%%%%%%%%%%%%%%%%%%%%%%%%%%%%%%%%%%%%%%%%%%%%%%%%%%
\begin{figure}
   \centering 
\includegraphics[width=0.525\textwidth]{FIGURES/Sensitivity.pdf}
\caption{The LISA sensitivity curve with the strain due to two massive black holes each with a mass
  of 5000 \msolarc. The merger of two 5000~\msolar black holes at $z = 15$ gives a
  signal to noise ratio of 33 that is visible by LISA. The inspiral will enter the LISA band at
  approximately $5 \times 10^{-4}$ Hz, corresponding to a time to merger of approximately 30 months.}
\label{Fig:Sensitivity}
\end{figure}
%%%%%%%%%%%%%%%%%%%%%%%%%%%%%%%%%%%%%%%%%%%%%%%%%%%%%%%%%%%%%%%%%%%%%%%%%%%%%%%%%%%%%%%%%%%%%%%%%%%%%%%%%
%====================================================================
\section*{Acknowledgments}
%====================================================================

\noindent JR acknowledges support from the Royal Society and Science Foundation Ireland under
grant number URF$\backslash$R1$\backslash$191132. JHW is supported by National Science Foundation grants AST-1614333 and
OAC-1835213, and NASA grants NNX17AG23G and 80NSSC20K0520.  
BWO acknowledges support from NSF grants PHY-1430152, AST-1514700, AST-1517908, and  OAC-1835213, by NASA grants NNX12AC98G and NNX15AP39G, and by HST-AR-13261 and HST-AR-14315.
JR wishes to acknowledge the DJEI/DES/SFI/HEA Irish Centre for High-End Computing (ICHEC) for the
provision of computational facilities and support on which the zoom simulations were run.
The original Renaissance simulations were performed on Blue 
Waters, which is operated by the National Center for Supercomputing Applications (NCSA)
with PRAC allocation support by the NSF (awards ACI-1238993, ACI-1514580, and OAC-1810584).
This research is part of the Blue Waters sustained-petascale computing project, which
is supported by the NSF (awards OCI-0725070, ACI-1238993) and the state of
Illinois. Blue Waters is a joint effort of the University of Illinois at
Urbana-Champaign and its NCSA.  The freely available plotting library {\sc
matplotlib} \citep{matplotlib} was used to construct numerous plots within this
paper. Computations and analysis described in this work were performed using the
publicly-available \enzo{}\citep{Enzo_2014, Enzo_2019} and \yt{} \citep{YT} codes,
which are the product of a collaborative effort of many independent scientists
from numerous institutions around the world. Their commitment to open science
has helped make this work possible.


\label{lastpage}
\bibliographystyle{mn2e}
\bibliography{mybib}
\end{document}


